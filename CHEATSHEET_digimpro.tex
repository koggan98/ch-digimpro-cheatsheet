% Options for packages loaded elsewhere
\PassOptionsToPackage{unicode}{hyperref}
\PassOptionsToPackage{hyphens}{url}
%
\documentclass[
]{article}
\usepackage[fontsize=8pt]{fontsize}

\usepackage[a4paper, margin=0.5in]{geometry}
\usepackage{lmodern}
\usepackage{amssymb,amsmath}
\usepackage{ifxetex,ifluatex}
\ifnum 0\ifxetex 1\fi\ifluatex 1\fi=0 % if pdftex
  \usepackage[T1]{fontenc}
  \usepackage[utf8]{inputenc}
  \usepackage{textcomp} % provide euro and other symbols
\else % if luatex or xetex
  \usepackage{unicode-math}
  \defaultfontfeatures{Scale=MatchLowercase}
  \defaultfontfeatures[\rmfamily]{Ligatures=TeX,Scale=1}
\fi
% Use upquote if available, for straight quotes in verbatim environments
\IfFileExists{upquote.sty}{\usepackage{upquote}}{}
\IfFileExists{microtype.sty}{% use microtype if available
  \usepackage[]{microtype}
  \UseMicrotypeSet[protrusion]{basicmath} % disable protrusion for tt fonts
}{}
\makeatletter
\@ifundefined{KOMAClassName}{% if non-KOMA class
  \IfFileExists{parskip.sty}{%
    \usepackage{parskip}
  }{% else
    \setlength{\parindent}{0pt}
    \setlength{\parskip}{6pt plus 2pt minus 1pt}}
}{% if KOMA class
  \KOMAoptions{parskip=half}}
\makeatother
\usepackage{colortbl}
\usepackage{xcolor}
\IfFileExists{xurl.sty}{\usepackage{xurl}}{} % add URL line breaks if available
\IfFileExists{bookmark.sty}{\usepackage{bookmark}}{\usepackage{hyperref}}
\hypersetup{
  hidelinks,
  pdfcreator={LaTeX via pandoc}}
\urlstyle{same} % disable monospaced font for URLs
\usepackage{longtable,booktabs}
% Correct order of tables after \paragraph or \subparagraph
\usepackage{etoolbox}
\makeatletter
\patchcmd\longtable{\par}{\if@noskipsec\mbox{}\fi\par}{}{}
\makeatother
% Allow footnotes in longtable head/foot
\IfFileExists{footnotehyper.sty}{\usepackage{footnotehyper}}{\usepackage{footnote}}
\makesavenoteenv{longtable}
\setlength{\emergencystretch}{3em} % prevent overfull lines
\providecommand{\tightlist}{%
  \setlength{\itemsep}{0pt}\setlength{\parskip}{0pt}}
\setcounter{secnumdepth}{-\maxdimen} % remove section numbering

\author{}
\date{}

\begin{document}

\renewcommand{\arraystretch}{1.73}
\arrayrulecolor{gray}
\begin{longtable}[]{p{0.04\textwidth}p{0.29\textwidth}p{0.24\textwidth}p{0.38\textwidth}}
\toprule
\textbf{Symbol}
 &
\textbf{Name}
 &
\textbf{Formula}
 &
\textbf{Description / Example}
\tabularnewline
\midrule
\endhead
\multicolumn{4}{c}{\textbf{Homography}}
\tabularnewline\hline
\(\tilde{X_i^c}\)
 &
Image coordinates
 &
\(\tilde{X_i^c} \backsim\ \tilde{H_b^c} \cdotp \tilde{p}_i^b\)
 &

\tabularnewline\hline
\(K_c\)
 &
Calibration matrix, Camera matrix
 &
\(K_c = \begin{bmatrix} f_x & 0 & c_x \\ 0 & f_y & c_y \\ 0 & 0 & 1 \end{bmatrix}\)
 &
If \(c_x=c_y=0\) the camera produces centerd images
\tabularnewline\hline
\(\tilde{x}\)
 &
homogenous transformation
 &
\(\tilde{x}= K_c*\begin{bmatrix} x\\y\\z \end{bmatrix}\)
 &
for direction of light rays f.e.
\tabularnewline\hline
\(\tilde{p}_i^b\)
 &
Object point
 &

 &
Coordinates on the planar object

\tabularnewline\hline
\(E_b^c\)
 &
Camera extrinsic matrix
 &
\(E_b^c = \begin{bmatrix} R_b^c & t_{cb}^c \\ \begin{bmatrix} 0 & 0 & 0 \end{bmatrix}& 1 \end{bmatrix}\)
 &
Relative pose between object and camera
\tabularnewline\hline
\(E_c^g\)
 &
Inverse rigid motion matrix
 &
\multicolumn{2}{l}{
\(E_c^g=(E_g^c)^{-1}=\begin{bmatrix} R_g^c&t_{cg}^c\\\vec{0}&1 \end{bmatrix}=\)
\(\begin{bmatrix} R_c^g&-R_c^gt_{cg}^c\\\vec{0}&1 \end{bmatrix}\)}
\tabularnewline\hline
\(R_b^c\)
 &
Rotation matrix
 &
\(R_b^c=\begin{bmatrix}r_x & r_y & r_z\end{bmatrix}\)
 &

\tabularnewline\hline
\(t_{cb}^c\)
 &
Translation vector
 &
\(t_c^b=p^c-R_b^c*p^b\)
 &

\tabularnewline\hline
\(\vec{E}_g\)
 &
Coordinate frame basis
 &
\(\vec{E}_g = \begin{bmatrix} \vec{e}_{g,x} & \vec{e}_{g,y} & \vec{e}_{g,z} \end{bmatrix}\)
 &
A coordinate frame consists of a basis and and an origin
\tabularnewline\hline
\(\vec{o}_g\)
 &
Origin of the coordinate frame
 &
\(\vec{o}\)
 &

\tabularnewline\hline
\(\vec{p}\)
 &
Point in the coordinate frame
 &
\(\vec{p} = \vec{e}_{g,x} \cdotp p_x^g + \vec{e}_{g,y} \cdotp p_y^g + \vec{e}_{g,z} \cdotp p_z^g + \vec{o}_g\)
 &

\tabularnewline\hline
\(\vec{p}\)
 &
Point in the coordinate frame
 &
\(\vec{p} = \vec{E}_g \cdotp p^g + \vec{o}_g\)
 &

\tabularnewline\hline
\(H_b^c\)
 &
Homography matrix
 &
\(H_b^c \backsim\ K_c \cdotp \begin{bmatrix} r^c_{b,x} & r_{b,y}^c & t_{cb}^c \end{bmatrix}\)
 &

\tabularnewline\hline
\(r_{b,x}^c\)
 &
Rotation vector x
 &
\(r_{b,x}^c = \begin{bmatrix} 0 & -t_{cb,z}^c & t_{cb,y}^c \end{bmatrix}\)
 &

\tabularnewline\hline

 &
Object point to image point
 &
\(\begin{bmatrix} x_{s,i} \\ y_{s,i} \\ 1 \end{bmatrix} = H_b^c \cdotp \tilde{p}_i^b\)
 &

\tabularnewline\hline
\multicolumn{4}{c}{\textbf{Hough~transform}}
\tabularnewline\hline
\(x_i, y_i\)
 &
Image space coordinates
 &
\(y_i = m \cdotp x_i + c \Leftrightarrow c = - m \cdotp x_i + y_i\)
 &
Converted to parameter space, lines
\tabularnewline\hline
\(\theta\)
 &
Angle of point
 &
\(\theta\)
 &
angle between \(x\) and line in parameter space
\tabularnewline\hline
\(\rho\)
 &
Proper Line Parametrization
 &
\(\rho = x \cos(\theta) + y \sin(\theta)\)
 &
length of line
\tabularnewline\hline
\(\rho\)
 &

 &
\(\rho =\begin{bmatrix}x \\ y\end{bmatrix}^t \cdot n\)
 &
Test if a point is on a line
\tabularnewline\hline
\(n\)
 &
normal vector
 &
\(n = \begin{bmatrix} \cos(\theta) \\ \sin(\theta)\end{bmatrix}\)
 &
Defined by the \(\theta\) angle
\tabularnewline\hline
\multicolumn{4}{c}{\textbf{RANSAC~probabilities}}
\tabularnewline\hline
\(\epsilon\)
 &
Probability of picking an outlier
 &
\(\epsilon = \dfrac{N_{outliers}}{N_{inliers} + N_{outliers}}\)
 &
with \(N\) = no of, \(s\) = points, \(n\) = no. of trials
\tabularnewline\hline

 &
probability of picking individual inlier
 &
\(p=1-\epsilon\)
 &

\tabularnewline\hline

 &
probability of picking \(s\) inliers in sequence
 &
\(p=(1-\epsilon)^s\)
 &

\tabularnewline\hline

 &
probability of not picking \(s\) inliers in sequence
 &
\(p=1-(1-\epsilon)^s\)
 &

\tabularnewline\hline

 &
probability of not picking \(s\) inliers in sequence of \(n\)
trials
 &
\(p=(1-(1-\epsilon)^s)^n\)
 &

\tabularnewline\hline

 &
probability of picking at least in one of \(n\) trials \(s\) inliers in
sequence
 &
\(p_{success}=1-(1-(1-\epsilon)^s)^n\)
 &
for lines 2 , for circles 3 points are needed
\tabularnewline\hline

 &
expected number of trials needed
 &
\(n=\dfrac{log(1-p_{success})}{log(1-(1-\epsilon)^s)}\)
 &

\tabularnewline\hline
\multicolumn{4}{c}{\textbf{Geometric transformation}}
\tabularnewline\hline
\(\tilde{x}\)
 &
Intersection of two lines
 &
\(\tilde{x}=\tilde{I_1}\times \tilde{I_2}\)
 &
cross product of two lines defines their intersection
\tabularnewline\hline
\(\tilde{I}\)
 &
two points lie on the line
 &
\(\tilde{I}=\tilde{x_1}\times \tilde{x_2}\)
 &
cross product of two points define their collective line
\tabularnewline\hline
\multicolumn{4}{c}{\textbf{Matrix basics}}
\tabularnewline\hline
\(E\)
 &
unit matrix
 &
\(\begin{bmatrix} 1&0&0\\0&1&0\\0&0&1 \end{bmatrix}\)
 &

\tabularnewline\hline
\(R^{-1}\)
 &
Inverse rotational matrix
 &
\(R^{-1} = R^T\)
 &

\tabularnewline\hline
\(R_x\)
 &
Rotational matrix around x
 &
\(R_x=\begin{bmatrix} 1&0&0\\0&\cos&-\sin\\0&\sin&\cos \end{bmatrix}\)
 &

\tabularnewline\hline
\(R_y\)
 &
Rotational matrix around x
 &
\(R_y=\begin{bmatrix} \cos&0&\sin\\0&1&0\\-\sin&0&\cos \end{bmatrix}\)
 &

\tabularnewline\hline
\(R_z\)
 &
Rotational matrix around x
 &
\(R_z=\begin{bmatrix} \cos&-\sin&0\\\sin&\cos&0\\0&0&1 \end{bmatrix}\)
 &

\tabularnewline\hline
\multicolumn{4}{c}{\textbf{Camera calculations}}
\tabularnewline\hline
\(F\)
 &
Focal length in {[}\(mm\){]}
 &
\(F=\dfrac{L*x_{chip}}{2*W}\)
 &
\(L=\) Length from sensor to object, \(W=\) Width from sensor to object,
all in {[}\(mm\){]}
\tabularnewline\hline
\(f_x\)
 &
Focal length in {[}\(\dfrac{pixel}{mm}\){]}
 &
\(f_x = \dfrac{F*x_{pixel}}{x_{chip}}\)
 &
\(x_{pixel}=\) pixel in \(x\)-direction \(x_{chip}=\) length of sensor
chip in \(x\)-direction, analog for \(f_y\) with \(y_{pixel}\) and
\(y_{chip}\)
\tabularnewline\hline
\(f_x, f_y\)
 &
Focal length (Assumption)
 &
\(f_x = f_y = f\)
 &
In some calcuations just assume: focal length is the same in both
directions
\tabularnewline\hline
\(c_x, c_y\)
 &
image center coordinates
 &
\(c_x=\dfrac{x_{pixel}}{2}\)
 &
optical axis pointing perpendicularly through sensor chip center, analog
for \(f_y\) with \(y_{pixel}\) and \(y_{chip}\)
\tabularnewline
\bottomrule
\end{longtable}

\end{document}
